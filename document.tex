\documentclass[a4paper]{article}
\usepackage[utf8]{inputenc}
\usepackage[T1]{fontenc}
\usepackage{hyperref}
\usepackage[francais]{babel}
\renewcommand{\contentsname}{Table des matières}

\begin{document}

\begin{titlepage}
    \begin{center}
		{\Large Souveraineté numérique~: Reconquérir et protéger}\\
		{\large\textit{Les 4 piliers d’une stratégie planifiée et intégrée}}\par
		\vspace{2cm}
		{\large X-Alternative \& Intérêt Général\par}
		{\large Juin 2021\par}
		\vfill
	\end{center}
	Cette note est le fruit d’une collaboration inédite entre le laboratoire d’idées Intérêt général et le collectif X-Alternative.\\ 
	\textbf{Pour citer cette note ~} \textit{Intérêt général \& X-alternative, Souveraineté numérique : Reconquérir et protéger - Les 4 piliers d’une stratégie planifiée et intégrée, juin 2021}

\end{titlepage} 

\newpage
\tableofcontents
\addcontentsline{toc}{section}{Introduction~: Numérique partout, souveraineté nulle-part}
\newpage

\section*{Introduction~: Numérique partout, souveraineté nulle-part}
Amazon, Google, Apple, Microsoft, Facebook, Twitter, Uber, etc. Que nous le voulions ou non, les algorithmes et les outils numériques ont envahi nos vies. Pour s’alimenter, se déplacer, s’informer, écouter de la musique, rencontrer des personnes, nombreuses sont les tâches élémentaires de notre vie quotidienne que des millions d’êtres humains délèguent à des acteurs du numérique. Pourtant, le capitalisme mondialisé, à la fois financiarisé et numérisé, ne s’encombre d’aucune limite démocratique ou de la part des États. Seuls les consommateurs ont quelques marges de manœuvre, comme abandonnées pour donner l’impression d’une certaine liberté. Et la liberté est d’abord question de souveraineté. Être souverain, c’est la condition première pour exprimer ses choix démocratiques et ne pas se voir imposer des cadres fixés par d’autres.\par

Plus largement, des pans entiers de nos sociétés sont structurés par le numérique~:
\begin{itemize}
\item Dans l'économie où tous les secteurs en dépendent désormais, de l'industrie à l'hôtellerie en passant par la mode et les transports~;
\item Dans nos vies personnelles (protection des données, droit à la déconnexion)~;
\item Dans la vie politique où journalistes, élus et militants communiquent via Twitter, où l'on a pu voir exploser des groupes Facebook Gilets Jaunes en quelques jours à plusieurs centaines de milliers de membres~;
\item Dans la recherche et l’innovation, où modélisation et simulation numérique permettent des avancées notables notamment sur l’étude de l’évolution du climat, la découverte de molécules médicamenteuses, la dispersion des polluants, les reconstitutions archéologiques, \ldots
\end{itemize}
Pour toutes ces raisons, maitriser la chaîne de production de la filière, ou au moins une part, est plus que jamais nécessaire.\par

Pourtant, les gouvernants semblent aveugles ou impuissants depuis les années 2000. En 2000, la France avait avec Alcatel, certes déclinant, un numéro un mondial des télécom. Aujourd'hui Nokia licencie les derniers ingénieurs télécom de l’ex-conglomérat pendant que le gouvernement se félicite de leur reprise partielle par Qualcomm, géant américain. 
De ces conglomérats, de notre système de recherche publique, de notre système de formation, on aurait pu faire des atouts majeurs. Il ne restera bientôt plus rien, laissant donc nos vies personnelles, notre économie et notre vie politique à la merci des écosystèmes asiatique et américain. 

\noindent\textbf{30 ans à laisser passer le train}\par

30 ans de passivité, d'aveuglement, de pleins pouvoirs à des actionnaires guidés par leurs intérêts privés et mondiaux auront amené à cette situation déplorable. La taille atteinte par les géants américains ou asiatiques, les montants colossaux de leurs investissements financés par la manne qu'ils prélèvent sur le reste du monde, rendent la compétition frontale avec eux difficile, voire impossible, y compris à l'échelle européenne.

Nos gouvernants se reposent alors sur les startups, hypothétiques entreprises innovantes qui seraient en mesure de venir créer de nouveaux marchés pour bâtir les « géants de demain ». Mais les startups sont gérées par des fondateurs et financées par des banques, des business Angels ou des fonds dont la priorité est la plus-value, c’est-à-dire leur intérêt monétaire direct. L'absence de stratégie ou même simplement de dispositifs protecteurs au niveau de l'état, jointe à l'aversion au risque des investisseurs français, à leur stratégie capitaliste nobiliaire, conduit ces startups à se faire coter à l'étranger, voire racheter pour l'équivalent d'une bouchée de pain si elles rencontrent le succès. Ces startups, étant de toute façon à la fin de la chaîne logistique du numérique, reposent sur des infrastructures, des matériels, des briques logicielles qui sont développées ailleurs, les rendant, même en cas de succès, dépendantes des écosystèmes US et Asie. 

Pendant ces trois décennies, ce sont les USA et la Chine qui se sont taillé la part du lion. La plupart des éléments fondamentaux d’une infrastructure réseau et de calcul sont conçus et fabriqués hors des frontières européennes. Les données essentielles aux fonctions régaliennes de l’État, mais aussi de la santé et des transports par exemple sont confiées à des entreprises extra-européennes. 

Il nous faut acter l’échec des politiques menées depuis trente ans : nous sommes en situation de dépendance quasi-totale. Pourtant, un écosystème numérique souverain est essentiel pour développer de nouveaux services publics, pour développer les outils de la rupture écologique, pour garantir la sécurité de nos infrastructures stratégiques ou encore pour assurer un accès des citoyens à un Internet qui soit respectueux de leurs droits fondamentaux. 

\noindent\textbf{Reprendre le contrôle}

S’il est devenu impensable de développer seul une pile matérielle et logicielle complète, il est irresponsable de n’en maîtriser plus qu’une portion si mince. À des degrés divers, les infrastructures stratégiques de défense ou d’énergie comme les communications grand public doivent reposer, du moins en partie, sur des solutions nationales ou a minima sur des solutions de partenariats internationaux. 

Le développement et la sécurisation de solutions indépendantes passeront nécessairement par une recherche publique renforcée, des investissements publics massifs, un protectionnisme assumé et la sécurisation des capitaux investis, via des prises de participations majoritaires dans certaines structures de droit privé. Certaines mesures envisagées dans ce document, comme la constitution de monopoles publics sur les réseaux ou le financement direct d’entreprises par l’État peuvent rentrer en conflit avec les traités européens\cite{interet2019europe}. 

Un écosystème numérique souverain repose sur certains éléments fondamentaux : le processeur (ou plus généralement les composants électroniques), le système d’exploitation et les compilateurs, le réseau et le cloud. Sans ces briques-là, aujourd’hui aux mains des USA ou de la Chine, nos éditeurs logiciels sont à la merci des plateformes sur lesquelles ils opèrent. Si le numérique était le ferroviaire, les USA et la Chine feraient les motrices, les caténaires, les gares, les rails et le ballast pendant que nos entreprises se feraient concurrence au wagon bar. Reconstruire un environnement souverain dans ces domaines suppose de l’humilité devant l’immensité de la tâche. Pourtant ce n’est pas insurmontable, à condition de sauver ce qui existe encore, d’investir intelligemment, de bâtir des partenariats bilatéraux stratégiques et d’accepter d’apprendre ou de réapprendre en faisant. Tout défi porte sa part de risque et d’erreurs. Mais les enjeux sont trop élevés pour abandonner notre souveraineté.

Enfin, rien de tout ça ne peut être porté sans intégrer avec rigueur les contraintes écologiques et climatiques. L’évolution de la consommation énergétique de l’ensemble de la filière numérique n’est pas soutenable\cite{ferreboeuf2019pour}. Il s’agit donc de définir quels biens sont nécessaires et s’il est soutenable de les produire~; certains seront donc amenés à diminuer. Ceci doit être un inventaire politique et donc démocratique. Cela ne peut pas se faire sans planification. Laisser le marché décider, c’est la garantie de la défaite. Pourtant, les gouvernants ont failli et qui pourrait faire confiance à ceux qui ont tout détruit ? Tout ce qu’on décrit passe par une refondation de l’État, en s’appuyant sur la détermination et les compétences dont nous sommes la preuve vivante de l’existence. 

Cette note adopte une vision systémique de la perte de souveraineté sur le secteur numérique pris dans son ensemble. Elle propose des solutions précises et intégrées afin de penser une filière de bout en bout. Reconstruire implique de proposer des solutions concrètes et planifiées pour les quatre piliers du secteur numérique~: matières premières, composants, réseaux et logiciels.

\section{Matières premières : assurer un approvisionnement viable et écologique}
\subsection{Situation}
Le terme « dématérialisation », très largement utilisé pour qualifier la transformation des opérations de transmission et stockage des données au moyen d’outils numériques et d’internet, est trompeur. Si le caractère apparent du courrier électronique est d’éviter l’usage de la matière-papier, il a fallu néanmoins mobiliser de grandes quantités de matériaux – et corrélativement d’énergie – pour que la transmission puisse se réaliser. Ce sont les matériaux constitutifs des terminaux de départ et d’arrivée (ordinateurs, smartphones, etc.), des réseaux de transmission de l’information (câbles, fibres optiques, etc.), des centres de stockage et de redistribution (datacenters). Un micro-ordinateur pèse environ 2 kg, mais il aura fallu utiliser environ de 600 kg de matières premières pour le construire. On estime que pour fabriquer un appareil électrique à forte composante électronique, il faut mobiliser 50 à 350 fois son poids en matières premières\cite{flipo2013face}. La fabrication d’une puce de 2 grammes implique le rejet d’environ 2kg de matières terrestres\cite{gossart2012impacts}.

La question de la quantité des matériaux constitutifs des machines et des réseaux numériques est directement liée à celle de leur qualité.  Outre les grands métaux tels que l’aluminium, le fer, le silicium, abondants dans la croûte terrestre, et le cuivre qui l’est moins, le numérique mobilise nombre de « petits métaux » ou « métaux technologiques » (tantale, gallium, germanium, etc.) et de terres rares (néodyme, dysprosium, etc.), dont la concentration ne s’exprime plus en pourcentages mais en parties par million. Il mobilise aussi des métaux précieux (or, argent, platinoïdes) dont la concentration dans la croûte terrestre s’exprime en parties par milliard. La miniaturisation des puces, mémoires et appareils numériques et l’accroissement de la capacité des machines informatiques et des réseaux n’ont pu se réaliser qu’au prix du recours à des métaux de plus en plus nombreux et rares. Un ordinateur en mobilisait une vingtaine dans les années 1980, mais plus de 60 aujourd’hui.

La maîtrise du numérique passe par celle de la chaîne d’approvisionnement en ces ressources terrestres, jugées critiques jusque dans les rapports de l’UE. Or leur extraction se trouve aujourd’hui concentrée dans un petit nombre de pays. La Chine produit 95\% des terres rares consommées dans le monde et plus de 50\% de 13 autres minerais, dont le gallium, l’indium et même le silicium, élément pourtant très abondant sur terre. L’Afrique du Sud concentre 70\% de la production de platine et plus de 80\% de celle des autres platinoïdes, le Chili 80\% de celle du niobium, la République démocratique du Congo deux tiers de celle du cobalt\cite{eucriticalrawmaterials}. Ce déséquilibre résulte de la mondialisation des échanges et de la circulation des capitaux~: l’extraction des ressources terrestres est privilégiée dans les régions du monde où son coût est le plus bas, du fait d’une concentration en minerai relativement plus élevée ou d’un faible coût du travail –-- souvent des deux à la fois. Elle résulte aussi de politiques délibérées d’abandon des activités extractives menées dans beaucoup de pays développés, notamment en Europe et tout particulièrement en France. 

L’extrême déséquilibre mondial de la production des métaux rares est une source potentielle de ruptures d’approvisionnement et de crises productives, voire de conflits. La République Populaire de Chine, par exemple, n’hésite pas à utiliser son quasi-monopole comme levier de négociation comme lors du conflit des îles Senkaku-Diaoyu ou en 2018, lors du bras de fer avec l’administration Trump. 

\subsection{Solutions}
Nous envisageons plusieurs solutions, éminemment complémentaires. Quelle que soit la solution envisagée, la France doit constituer des stocks stratégiques, limiter sa propre consommation et contrôler les exportations hors UE. Comme pour tout le reste de la note, ceci doit s’envisager dans le temps long d’une planification de rattrapage, à la manière de la stratégie chinoise, menée avec détermination depuis les années 80\cite{strategiechinoise}. 
\subsubsection{Mines}
L’ouverture de mines en Europe est ainsi devenue un véritable enjeu. Nous devons cartographier nos ressources et développer des moyens écologiques d’extraction.

Il ne s’agit pas de rêver d’extraire chez soi l’ensemble des matières premières indispensables au développement du numérique, mais de tenir une place dans la production mondiale des métaux. Ainsi la France dispose-t-elle de ressources en métaux dits « critiques », notamment dans le massif Central, le Massif Armoricain, les Pyrénées\cite{brgm2013}. On y trouve par exemple du tantale, qui compose les micro-condensateurs des circuits électroniques, de l’antimoine, qu’on trouve dans les puces, de l’indium, utilisé pour fabriquer les écrans tactiles. Certes, ces gisements ne sont vraisemblablement pas exploitables au même niveau de rentabilité que les standards mondiaux actuels. D’autant qu’ouvrir des mines en France suppose d’emporter le consentement des populations locales – « l’acceptabilité sociale » – ce qui rend indispensable la minimisation des nuisances écologiques liées à l’extraction minière et induit un surcoût par rapport aux régions du monde où la recherche de rentabilité peut primer sur le respect de l’environnement et du travail humain.
 
Cela nécessite une formation spécifique, notamment en toxicologie ou relative aux impacts environnementaux spécialisée en extraction minière. Le corps des mines pourrait y retrouver sa raison d’être. Ce savoir-faire acquis serait utilement exportable auprès de partenaires stratégiques tels que les pays scandinaves, l’Amérique du Nord, le Chili, l’Afrique du Sud, l’Inde, la Russie ou l’Australie.
 
Nul doute que les surcoûts d’exploration production seront importants, mais ces surcoûts ne sont-ils pas le nécessaire prix à payer pour acquérir, sinon une indépendance, à tout le moins des marges de manœuvre stratégiques dans l’approvisionnement en matières premières des filières du numérique ? Ne sont-ils pas le prix à payer pour sortir de l’hypocrisie coupable consistant à exporter chez les autres les nuisances des activités extractives indispensables à nos consommations tout en leur donnant « en même temps » des leçons d’écologie ?\footnote{Dans les années 1980, la raffinerie de Rhône-Poulenc (aujourd'hui Solvay) purifiait 50\% du marché mondial de terres rares. La première partie du raffinage a été sous-traitée en Chine pour ne pas avoir à gérer la pollution qu’elle générait. Aujourd’hui, on ne compte plus les reportages déplorant les dégâts environnementaux et humains dus à l’extraction et au raffinage des terres rares dans l’empire industriel du milieu.}

Il faut aussi voir qu’au niveau mondial, la croissance de la consommation des métaux associée aux transitions numériques et énergétiques bute sur la finitude des ressources. Ceci ne manquera pas de provoquer une hausse des prix dans ces matières à moyen terme. Quand on sait qu’il faut 10 à 20 ans entre le moment où on projette d’ouvrir une mine et celui où débute son exploitation, c’est sans délai qu’il conviendrait de lancer les projets.
\subsubsection{Recyclage}
En second lieu, il convient d’organiser une véritable industrie du recyclage, tout en ayant conscience des limites et obstacles qu’elle rencontre dans les conditions technico-économiques actuelles. D’une part, dans les composants électroniques, les petits métaux se présentent le plus souvent non pas à un état pur, mais sous forme d’alliages. Cela rend les opérations de recyclage complexes, énergétivores et onéreuses. Pour cette raison, seuls les « grands métaux » (fer, aluminium, plomb, cuivre) sont aujourd’hui recyclés\footnote{https://ecoinfo.cnrs.fr/2014/09/03/3-le-recyclage-des-metaux/}. D’autre part, le recyclage n’est que d’un faible secours en période de forte croissance~: les volumes de métal qu’on en retire correspondent à des niveaux de consommation passés donc inférieurs aux besoins actuels. 

Toujours est-il qu’on ne peut imaginer un avenir soutenable en continuant à jeter et donc disperser dans la nature une grande partie des matières premières contenues dans les machines et appareils numériques devenus obsolètes. Être responsable vis à vis des générations futures, c’est déjà collecter, stocker et conserver ces déchets qui pourront être les gisements de demain. C’est aussi préparer la faisabilité des recyclages et orienter la production vers des produits à plus grande durée de vie, réparables et recyclables, en investissant dans la recherche fondamentale sur les matériaux et dans la recherche-développement sur les produits et procédés, une politique certes étrangère aux actuelles logiques financières et concurrentielles court-termistes.

Cette politique exige des investissements dans la recherche scientifique afin d’améliorer les rendements de recyclage. En complément, nous pouvons légiférer quant à leur non-usage dans les autres secteurs (éoliennes, solaire, luminophores, etc.) afin de préserver ces ressources là où elles restent indispensables. Des barrières douanières, sujettes aux traités européens, pourraient également inciter les industriels à produire des appareils compatibles avec nos standards. 
Il faut également édicter des normes valables sur le territoire national. Les produits doivent être recyclables et réparables, dans une mesure décidée par l’Assemblée nationale, sous peine de subir des taxes voire des interdictions de commercialisation. Les composants doivent être interchangeables afin de garantir la réparation et leur pérennité dans le temps long.

Dans une optique de solidarité internationale, nous pouvons également proposer à des pays miniers ou producteurs une extraction qui pollue le moins possible et économiquement viable. Ceci pourrait s’articuler avec des accords de défense, de transferts technologiques ou des partenariats universitaires. La puissance des industriels miniers (Rio Tinto, BHP Billiton,\ldots) ainsi que la corruption endémique dans certains pays producteurs ne doit pas nous faire peur. Il faut tordre le bras aux premiers, via des compromis diplomatiques et le développement de notre propre filière industrielle, et contraindre les autres, via l’alliance avec les peuples locaux. 

\section{Composants \& processeurs~: concevoir localement}
Le processeur est l’élément central des ordinateurs, qu’ils soient de petites machines embarquées ou des supercalculateurs. En tant que tel il s’agit d’un enjeu de souveraineté majeur. La France doit disposer d’un accès à une technologie de processeur qui lui soit propre, ou \textit{a minima} codéveloppée avec ses voisins. De même, les composants électroniques, indispensables à l’ensemble de l’industrie, doivent être disponibles et reposer sur des technologies maitrisées. L’existence d’instructions non-documentées permettant de modifier le microcode des processeurs Intel\footnote{https://github.com/chip-red-pill/glm-ucode} nous montre, une fois de plus, que nous devons être capables de concevoir, produire et auditer des composants complexes. 

Nous pensons qu’il est fondamental de prendre le contrôle de quelques acteurs industriels via une participation majoritaire. La participation minoritaire associée à une \textit{golden share} n’offre pas de garantie suffisante de pérennité de l’activité (comme nous l’a montré l’affaire Alstom), et surtout ne nous donne pas le contrôle des directions stratégiques. Si nous voulons décider, en régime capitaliste il faut posséder. 

\subsection{Conception}
Nous proposons de nous baser sur ARM, société britannique à capitaux japonais qui conçoit des processeurs généralistes de toute gamme. Du smartphone aux supercalculateurs\footnote{https://top500.org/lists/top500/list/2020/06/}, ARM est un acteur incontournable dans le domaine et un des seuls dont le siège et les centres de recherche sont sur le territoire européen ou britannique. 
La commission européenne ne s’y est pas trompée quand elle a lancé « European Processor Initiative » un consortium européen visant à produire un processeur généraliste de haute performance d’ici 2021 et un supercalculateur \textit{exascale} d’ici 2022. La commission a doté ce consortium d’un milliard d’euros, en s’engageant à acheter le supercalculateur. Ce projet propose pour l’instant un noyau principal de type ARM, entouré de petits cœurs spécialisés en RISC-V\footnote{Reduced Instruction Set Computer (Jeu d’instructions réduit)} ou des cœurs dédiés externes comme le MPPA\footnote{Massively Parallel Processor Array} de Kalray. L’objectif est d’avoir un produit similaire aux processeurs Epic de AMD (x86), dans une architecture auditable et interopérable avec des composants tiers. La production du processeur serait confiée à TSMC\footnote{Taiwan Semiconductor Manufacturing Company}, groupe taiwanais leader mondial dans le domaine. 

En 2016, ARM a été racheté par SoftBank, un investisseur financier japonais. Celui-ci a effectivement laissé le siège à Cambridge et n’a pas touché aux centres de R\&D, ce qui permettait à la commission de continuer à considérer ARM comme une entreprise européenne sur laquelle s’appuyer. 

ARM est actuellement mis en vente par SoftBank et l’acquéreur est NVIDIA, géant américain du secteur. SoftBank a accepté l’offre qui doit maintenant être validée par les différents régulateurs du marché. Au Royaume-Uni, on s’alarme de la situation en rappelant qu’un investisseur industriel américain finira par rapatrier le siège et les centres de R\&D aux USA tout en soumettant l’entreprise aux lois CFIUS\footnote{https://home.treasury.gov/policy-issues/international/the-committee-on-foreign-investment-in-the-united-states-cfius/cfius-laws-and-guidance} et en détruisant sa neutralité vis-à-vis des clients européens ou asiatiques, parfois concurrents de NVIDIA. 

Ce serait la mort dans l’œuf du projet européen et au-delà, la perte de compétences fondamentales dans la conception de processeurs qui se dessinerait à court-terme. En matière de souveraineté numérique comme ailleurs, la priorité est donc de sauver le déjà-là.

Il faut racheter l’entreprise. Le temps de validation par les autorités de régulation, nous laisse une fenêtre permettant de monter une contre-offre à composante française. Celle-ci devrait être montée conjointement par la France et le Royaume-Uni. En effet, l’ADN de ARM est à Cambridge et la société ne pourra fonctionner qu’en gardant une part de son identité britannique. 

Le montant évoqué est de 40 milliards. Une option est de réunir l’intégralité de la somme via l’endettement public en se rémunérant sur les résultats futurs. Une autre est de monter un fonds souverain, financé par une réorientation d’une partie de l’épargne des particuliers, quelques investisseurs institutionnels et une part de dette souveraine ad hoc. En cas d’insuffisance, il suffirait d’en réunir la majorité par les deux États, et d’introduire une partie minoritaire en bourse. Les places financières de Londres et Paris sont sans doute capables de mener cette opération. 

ARM vend ses modèles sous licence à de nombreux partenaires industriels, notamment américains et asiatiques. Ceux-ci, dont Microsoft, Qualcomm et Apple, ont approché les régulateurs européens pour ralentir la vente. Afin de leur garantir la continuité de leurs opérations, il peut être envisagé de leur céder chacun une part très minoritaire du capital.

Par ailleurs, fort de ses capitaux et de ses revenus, ce fonds souverain pourrait financer des recherches sur de nouvelles architectures (RISC-V) et de nouveaux modèles de calcul (potentiellement hors IEEE-754 \cite{hunhold2017unum}, quantique ou autres) ainsi que leurs mises en pratique industrielles dans une optique de long-terme. Qu’il s’agisse du plan A ou de la solution envisagée comme plan B, il nous faudra alors pousser vers les logiques \textit{open-hardware}\footnote{Matériel libre de droit : transposition au matériel des principes du logiciel libre.}, aujourd’hui dominées par les géants privés du secteur. L’engagement financier de la France devra être à la hauteur, les pays de l’Europe de l’Est ayant une coloration atlantiste plus forte. 

\subsection{Fabrication}
Le plan de relance américain prévoit 50 milliards de dollars pour l’industrie des semi-conducteurs  pendant qu’Intel investit 20 milliards dans des nouvelles usines de puces. De son côté, la Chine poursuit une stratégie d’autonomisation en empilant les plans quinquennaux. La compétition entre ces deux blocs laisse la France et l’Europe avec elle sur le banc des spectateurs. Sans viser une position hégémonique, nous devons maitriser une part de la fabrication de puces, en envisageant une montée en gamme progressive. 
Via un mécanisme similaire à celui utilisé pour racheter ARM, les États français et italien investissent chacun quelques 3 milliards d’euros pour prendre le contrôle de STMicroelectronics, via la holding commune qui en détient déjà 25\%. 

Les 25\% d’ARM et les 25\% de STMicroelectronics détenus par la France sont alors placés dans une structure commune, chargée de coordonner conception et fabrication, et de servir au mieux les besoins des industries françaises, sans pour autant porter préjudice aux relations entre ARM et ses autres partenaires industriels. 

Des fonds sont engagés pour les fonderies existantes de STMicroelectronics afin qu’elles puissent produire par elles-mêmes une version du processeur en lithographie plus grossière (16-32nm), notamment pour le marché de l’embarqué, de l’\textit{IoT} et les besoins de petites et moyennes séries des PME françaises et européennes.

Le partenariat avec TSMC est maintenu, sous conditions d’investissements en France, en Italie et au Royaume-Uni – par exemple une petite unité de production pour alimenter le marché intérieur des supercalculateurs. L’objectif est de pouvoir graver la puce en lithographie optimale (3-5nm) pour les applications haute-performance. 

En parallèle, une filière de recherche fondamentale associant étroitement universités, CNRS, grandes écoles et industrie du semi-conducteur doit nous permettre de rattraper le retard accumulé sur les technologies de gravure, afin de pouvoir disposer à moyen terme d’une solution industrialisable complémentaire à TSMC et aux technologies EUV d’ASML.  
C’est la stratégie employée par la Chine face aux mesures d’embargo décidées par les USA. Encore une fois, il ne s’agit pas de partir de zéro, ni de concurrencer frontalement TSMC, les deux stratégies sont vouées à l’échec. Il s’agit d’une troisième voie, où l’on capitalise sur l’existant et on accepte de \emph{perdre de l’argent} en apprenant et en faisant nous-même, sans pour autant tuer toute collaboration avec les géants du secteur, ce qui n’aurait pour effet que de nous isoler.

Les \textit{smartphones}, les ordinateurs, leur assemblage et leurs interfaces utilisateurs sont des composants majeurs du secteur numérique. Ils ne sont pas traités dans le détail par cette note en raison des potentialités ouvertes par le reste de l’écosystème proposé ici. Le sujet est vaste et mérite un traitement à part entière.

\section{Réseau~: un service public de l’accès à internet}
Internet étant devenu indispensable à la vie commune, au même titre ou presque que l’eau ou l’électricité, l’accès au réseau est un droit fondamental et doit reposer sur un service public. 
\subsection{Situation}
Le développement de l’internet grand public ayant suivi de près l’ouverture du marché, il serait tentant de mettre au crédit de la concurrence les deux décennies d’innovation technique que nous venons de vivre. C’est en réalité plus compliqué que ça. Le fonctionnement monopolistique du service public avait en son temps permis le raccordement de tous les foyers français au téléphone et à la France d’être à la pointe de l’innovation. Quant aux protocoles à l’origine d’Internet et des technologies d’accès, ils sont la plupart du temps le fruit de la recherche publique, des universités américaines et européennes ou des opérateurs publics nationaux. 

Il n’est même pas évident qu’on puisse dire que la baisse des prix soit une conséquence de la concurrence, il suffit de se rappeler des condamnations pour entente des opérateurs mobiles au début des années 2000. La concurrence a pu permettre des innovations comme la Freebox mais elle a aussi entraîné de nombreux surcoûts. Au-delà des coûts structurels de marketing, de gestion, ou de rémunération du capital, il s’est agi d’encourager les opérateurs à déployer leur propre réseau. Il existe aujourd’hui de multiples infrastructures réseaux fixes et mobiles, déployées en parallèle par chaque opérateur pour offrir ses services. Cela engendre un coût financier et écologique important, en plus d’importantes contraintes d’interconnexion.  

Par ailleurs, la privatisation de ce secteur comporte aussi un enjeu social, notamment pour l’égalité d’accès aux services de télécommunication. On se souviendra de la « bataille du téléphone » menée par Giscard d’Estaing dans les années 70 pour permettre à la France de rattraper son retard en passant de 6 à 20 millions de lignes en quelques années, de se doter d’un des meilleurs réseaux téléphoniques du monde et de faire de Thomson et CGE (futur Alcatel) des acteurs majeurs du secteur. Ceci ne fut possible que sous l’impulsion d’une véritable planification politique financée par le public.  

N’ayant plus aujourd’hui d’opérateur national nous ne pouvons que déplorer les zones blanches de la téléphonie mobile ou du très haut débit, zones non équipées car non rentables pour les opérateurs privés. C’est alors de nouveau au public de pallier les déficits d’investissement des acteurs privés par des montages complexes et coûteux de RIP (Réseaux d’Initiative Publique) et autres PPP (Partenariat Public Privés). Ainsi c’est ici aussi la logique habituelle du néolibéralisme qui s’est mis en place : privatisation des profits (dans les zones rentables) et socialisation des pertes (dans les zones non rentables).

On notera au passage une nette dégradation de la qualité et de la pérennité des infrastructures déployées. Alors que le réseau mis en place sous l’impulsion de l’État était d’une qualité mondialement reconnue, les points de mutualisation du FTTH\footnote{Fiber to the Home, ce qui signifie « Fibre optique jusqu'au domicile »} sont dans un état dégradé. Comme dans l’ensemble de l’industrie, la logique a été à la chasse aux coûts et au recours massif une sous-traitance pressurée, conduisant à des réseaux dont on pourra mesurer la mauvaise qualité dans les prochaines décennies.

\subsection{Solutions}
Nous proposons deux solutions, à envisager ensemble. Elles sont possibles (au moins partiellement) dans les structures légales et internationales existantes, à condition d’en avoir la volonté politique.
\subsubsection{Imposition}
Il est indispensable de redéfinir le périmètre financier taxable des géants de l’internet afin de les faire contribuer. Cela suppose d’engager le bras de fer avec Washington, ces compagnies supranationales étant pour l’essentiel américaines. Mais cela suppose aussi de tordre le bras de Bruxelles et probablement des traités européens. Ceux-ci permettent en effet à ces géants de déclarer des bénéfices quasi-nuls en France (et ailleurs en Europe) par jeu d’écriture comptable via les \textit{usual suspects} (Luxembourg, Pays-Bas, Irlande).
 
Pour autant, si symbolique qu’elle soit, l’efficacité économique d’un tel effort fiscal et diplomatique resterait marginale. En effet le chiffre d'affaires de Google France était d’un peu moins de 500 millions d’euros en 2019. Netflix a un chiffre d'affaires mondial de 25 milliards de dollars en 2020, soit moins d’un milliard en France en comptant que le PIB français est de 3.3\% du PIB mondial. En comparaison, le chiffre d'affaires total des opérateurs français est de l’ordre de 30 milliards d’euros par an. Ainsi, pour indispensable qu’elle soit, la taxation des géants du web ne représente pas une source de financement suffisante pour financer les infrastructures. Dans tous les cas, il ne s’agirait pas d’une solution suffisante, d'autant qu'elle ne changerait pas les structures ni les rapports de force.

\subsubsection{Nationalisation et rationalisation}

Afin d’uniformiser le réseau internet français, aujourd’hui éclaté en une myriade d’acteurs privés ou publics, nous proposons ainsi de nationaliser toutes les infrastructures filaires ou sans fil et de les regrouper au sein d’un acteur unique, entièrement détenu par l’État. Les traités européens interdisent en effet d’avoir un FAI (Fournisseur d’accès internet) unique et statuent qu’ils doivent opérer en concurrence. En attendant de pouvoir renationaliser Orange et d’absorber les parties françaises de ses concurrents, nous proposons donc que les FAI opèrent sur réseau entièrement public, en complément d’un FAI « social » opérant à des tarifs réglementés. L’utilisation du réseau par les FAI fera l’objet d’une tarification permettant de financer les infrastructures sur l’intégralité du territoire national et le FAI « social ». Coup de chance, les opérateurs sont en train de vendre leurs infrastructures à des pure-players ou des fonds d’investissement étrangers afin de transformer leurs charges fixes en loyers récurrents. Le rachat de leurs infrastructures par l’État devrait donc leur convenir pendant que l’Etat contrôlerait l’infrastructure, et donc \textit{in fine} le comportement du marché privé.

Par ailleurs, les interconnexions terrestres ou sous-marines en fibre optique sont aujourd’hui une brique fondamentale de l’économie mondiale. Orange est souvent actionnaire minoritaire des sociétés d’exploitation de ces câbles, souvent déployés par des consortiums. Depuis quelques années, face à la croissance du volume de données et l’insuffisance des investissements public ou des FAI, les géants de l’internet commencent à tirer leurs propres câbles. Au minimum, il faut conditionner l’atterrissage des câbles sur le territoire national à la présence de notre opérateur souverain au conseil au conseil d’administration des sociétés d’exploitation de ces câbles privés. En être automatiquement actionnaire à 51\%, à la manière de ce que fait la Chine pour autoriser la pénétration de leur marché national nous paraît raisonnable. Une des filiales de Nokia, \textit{Alcatel Submarine Networks}, est par ailleurs un des grands acteurs du domaine qui devra être intégrée dans la stratégie globale, voire intégrée dans le monopole public du réseau physique évoqué \textit{supra}.

\subsection{Composants réseau}
Par ailleurs, les composants réseaux, sont des briques indispensables au déploiement d’un réseau sécurisé et efficace. Il reste tout à fait envisageable d’en acheter à l’étranger, notamment pour les grands volumes, mais nous ne pouvons nous résoudre à perdre complètement le savoir-faire, comme la destruction de Nokia le laisse présager. La fenêtre de sauvegarde de l’outil de R\&D de Nokia est fermée. Nous devons néanmoins récupérer les savoir-faire dans des unités de recherche et de production et remonter un acteur industriel du réseau civil. La nationalisation des infrastructures évoquées ci-dessus nous donne un client massif et un employeur tout trouvé pour une filière de R\&D. 

\section{Logiciel~: pour des outils libres, sûrs et souverains}
En matière de logiciels, plusieurs axes sont identifiables : les systèmes d’exploitation et compilateurs, le cloud, la bureautique, la recherche sur internet, et enfin l’intelligence artificielle. La chance que nous avons en matière logicielle, comparativement au matériel, est de disposer des énormes efforts menés par la communauté du libre depuis les années 80. Les chantiers sont immenses et il est important de procéder graduellement, par empilement de plans pluriannuels étroitement articulés entre eux.
 
La commande publique, comme marché massif, jointe à la commande subventionnée des PME permettra de financer les développements tout en nous libérant petit à petit de la dépendance aux tiers. La publicité des systèmes réalisés pourrait se faire via les services publics actuels. Par exemple, nous proposons d’intégrer Qwant, moteur de recherche souverain (cf. \textit{infra}), à l’ensemble des sites administratifs, de les héberger chez OVH sur des serveurs souverains, et de le faire savoir, y compris à l’international et dans les médias. 
Plus généralement, il s’agit également d’intégrer des compétences au sein des administrations et d’en faire un outil de reconquête via la commande et l’investissement publics. 

\subsection{Systèmes d'exploitation, compilateurs, bureautique}
Du système d’exploitation à la bureautique, nous proposons de transformer le « Socle interministériel de logiciels libres » en « Pôle public du logiciel », fort d’ingénieurs fonctionnaires, issus de corps techniques à haute qualification, capables de fournir les développements dont nous avons besoin. 

Ce pôle public du logiciel développera principalement une solution Linux qui aura vocation à être distribuée à l’ensemble de la fonction publique d’État et territoriale (en procédant par étapes). Cela suppose d’avoir des ingénieurs respectivement en charge~:
\begin{itemize}
\item Du noyau Linux~;
\item Des compilateurs (LLVM / GCC)~;
\item De la distribution (frubuntu ?)~;
\item Des logiciels bureautiques~;
\item Du portage d’applications tierces.
\end{itemize}
En cohérence avec la prise de contrôle de l’infrastructure matérielle, le développement dans le noyau et dans les compilateurs sont inévitables et à réaliser en coopération avec les entreprises nouvellement publiques ou parapubliques.

Par ailleurs, déployer telle quelle une solution libre en remplacement d’une solution Windows-Office n’est pas à portée immédiate et il y a beaucoup à apprendre de l’expérience munichoise qui a eu lieu de 2004 à 2017. Il faudra naturellement apprendre à porter des applications, convertir des documents, des bases de données, ou fournir des outils de compatibilité.

Une fois viabilisée, cette solution pourrait être ensuite distribuée commercialement (pensons au protectionnisme solidaire) par un OEM comme LDLC en vue de pénétrer petit-à-petit (et sous forte incitation tarifaire) le parc privé.

En complément, ce pôle public du logiciel assurera, joint notamment à l’ANSSI et aux autres administrations de l’État, une capacité d’audit des logiciels déployés dans l’administration, ou, sur requête motivée, des acteurs privés.

\subsection{Mutualisation pour les administrations et les PME}
Nombre de nos administrations et PME fonctionnent aujourd’hui avec des logiciels obsolètes, issus de contrats de prestations mal suivis avec des interlocuteurs inexistants (pour cause de faillites ou de rachats successifs). La gestion de ces contrats de support a été laissée à l’abandon, laissant des CHU, des universités ou des PME industrielles avec des logiciels dysfonctionnels fossilisés sur du matériel lent. 

Les directions de ces établissements, poussés par la logique austéritaire, englués dans les règles absconses des marchés publics, diffèrent ou annulent des investissements nécessaires, parfois pour des montants dérisoires. Souvent la mise à jour des logiciels est effectivement très complexe, et les DSI (s’il elles existent) se retrouvent en position de forte vulnérabilité face à des prestataires privés fortement concentrés qui peuvent facturer à volonté. De même, la rareté de compétences relatives au logiciel dans ces secteurs place les acheteurs en grande position de faiblesse. 

Afin de rééquilibrer le rapport de force afin de rééquiper nos administrations et notre tissu de PME, nous proposons de transformer la Direction Interministérielle du Numérique (DINUM). La DINUM, aujourd’hui organe de recommandation sans compétence technique approfondie au niveau de la direction pourrait devenir la base d’une nouvelle Direction des services informatiques de l’État, en charge de~:
\begin{itemize}
	\item Recueillir les besoins des différentes administrations publiques ;
	\item Proposer des solutions, en concertation avec le Pôle Public du Logiciel ;
	\item Déployer les systèmes et assurer la maintenance opérationnelle des logiciels mutualisés ;
	\item Coordonner les codéveloppements entre tiers, Pôle Public du Logiciel et administration cliente ; 
	\item Mutualiser les contrats avec des tiers (quand il y a lieu), et en assurer le suivi.
\end{itemize}
Des essais ont été menés afin de centraliser ce type de besoins. Elles se sont heurtées à des incompétences de la haute hiérarchie et des intérêts éclatés voire personnels. À ce titre, le projet Louvois de paiement des soldes militaires est un exemple de ce qu’il ne faut pas faire. La réussite de cette réforme suppose une refonte des liens de l’État avec les acteurs privés. Sans prétendre centraliser l’ensemble des besoins des administrations, il s’agit de le faire sur les logiciels de cœur (matériel, système d’exploitation, bureautique, paye, comptabilité notamment) et leur support utilisateurs. L’État, via cette DSI, serait ainsi un interlocuteur unique à même de rééquilibrer le rapport de force avec le privé et permettre notamment d’appuyer techniquement les collectivités locales et administrations de Sécurité sociale. 

Toutefois, il faut rester humbles, surtout quand on part de pas grand-chose. Commençons petit, par des choses faisables avant de nous attaquer aux grands problèmes complexes. Ne multiplions pas les acteurs, mais gardons une maitrise d’ouvrage centralisée, une maitrise d’œuvre unique (le ministère concerné) et un nombre limité de prestataires. 

\subsection{Recherche Internet, \textit{Cloud}}
Il existe déjà un moteur de recherche français qu’il s’agit d’appuyer et renforcer~: Qwant, est détenu à 35\% par la caisse des dépôts et consignations qui y dispose de droits de vote doubles. Pour quelques dizaines de millions de plus, la France pourrait en prendre le contrôle complet et diriger la stratégie et les investissements plutôt que la laisser à des actionnaires dont les intérêts divergent de ceux de la Nation\footnote{Il faut noter ici que la CDC doit subir une profonde réforme quant à son fonctionnement : elle doit sortir de la logique actionnariale pour devenir un des bras financiers de la stratégie industrielle de l’Etat.}. Il est déjà déployé par défaut sur les postes de certaines administrations en vue d’augmenter sa part de marché et donc ses retours utilisateurs.

Nous proposons d’y investir davantage, afin d’assurer un service de qualité, compétitif avec les standards du marché sur un périmètre restreint. D’ici quelques années, il serait proposé comme moteur de recherche par défaut dans la version du navigateur évoquée précédemment. En parallèle, la loi devra imposer aux navigateurs et systèmes d’exploitation d’en proposer l’installation, en mettant un réglage « facile d’accès » que la jurisprudence pourra se charger de définir. 

De même, la France dispose également de fournisseurs \textit{cloud}~: OVHCloud est de loin le plus important d’entre eux, présent dans 19 pays avec un demi-milliard de chiffre d’affaires. Une première étape est de lui attribuer les contrats d’hébergement de données publiques. Il y a deux écueils à éviter toutefois~:
\begin{itemize}
\item Le droit européen des marchés public impose des règles qui empêchent les acteurs français (ou européens) de se voir attribuer des marchés\footnote{https://www.legifrance.gouv.fr/codes/id/LEGIARTI000037703671/2019-04-01/}~: « l'acheteur garantit aux opérateurs économiques [\ldots] issus des États parties à l'accord conclu dans le cadre de l'OMC [\ldots], un traitement équivalent à celui garanti à ceux [\ldots] issus de l'Union européenne ». En effet, tous les 5 jours, Amazon dépense en R\&D l’équivalent du chiffre d’affaires d’OVHCloud. Comment être concurrentiel dans ces conditions ? 
\item À défaut de sortir de l’UE (ce qui doit rester une option sur la table) ou d’en renégocier les traités (ce qui est probablement impossible), il est possible d’écrire des appels d’offres avec des clauses qui rendent obligatoire le choix de tel ou tel acteur. Les USA ne se gênent pas, et beaucoup d’autres pays européens non plus. 
\item À supposer qu’on puisse développer OVHCloud en lui attribuant des contrats d’hébergement malgré la concurrence et le droit européen, il resterait dangereux de dépendre d’un seul acteur dans le domaine, surtout si celui-ci reste privé. Il faudrait alors faire émerger d’autres acteurs, déjà existants aujourd’hui, à qui l’on attribuerait d’autres contrats.
\end{itemize}
La France s’appuie déjà sur l’expertise de l’ANSSI, l’agence nationale de la sécurité des systèmes d’information. Nous avons également une école de cryptographie reconnue qui repose sur l’excellence de la formation en mathématiques. En souffrance comme l’université ou la recherche publique, l’école de formation française en informatique théorique, notamment en matière de sécurité, reste capable du meilleur et ne nécessite pas d’autre effort qu’un retour aux principes fondamentaux~: des scientifiques qui cherchent en indépendance avec des moyens pour le faire. Par ailleurs le tissu industriel, quoique menacé, est encore solide avec des grands groupes comme Thalès ou des jeunes entreprises prometteuses, dont il nous faut garantir la pérennité et éviter le rachat par de grands groupes américains de certaines « pépites ».

L’acquisition d’ARM (et le contrôle partiel des FAB –-- voir \textit{supra}) permettrait d’avoir une solution auditable dans les processeurs et donc de nous soustraire aux potentielles portes dérobées. Le renforcement de notre participation dans STMicroElectronics servirait ce même objectif dans les composants embarqués. De même, en récupérant les capacités techniques dont Nokia se sépare (via Thalès par exemple), nous pourrions élaborer nos propres composants réseaux pour les applications de faible volumétrie et de criticité importante, tout en gardant une capacité d’audit des matériels importés. 

À l’échelle internationale, nous devons continuer à plaider pour l’intégration de tous les organismes normalisateurs américains (NIST, ICANN, certificats racines, etc.) au sein de l’ONU et en faire des organisations non lucratives et indépendantes des États. La puissance américaine ne cèdera pas ses prérogatives sans lutter mais seule l’ONU et le multilatéralisme ont la légitimité pour réguler le bien public mondial qu’est internet.

\addcontentsline{toc}{section}{Conclusion~: Serviles ou maîtres de notre destin ?}
\section*{Conclusion~: Serviles ou maîtres de notre destin ?}
Le chantier pour recouvrer la souveraineté numérique est immense, la situation s'étant aujourd'hui considérablement dégradée suite aux trahisons et sabotages commis par les tenants de l'idéologie néolibérale et les acteurs du profit privé. Le préalable à tout rebond est l'existence d'une volonté politique forte, comparable à celle des forces qui résistèrent à la capitulation et à l'indignité Vichyste en 1940, à celle du CNR pour la reconstruction du pays en 1945. À l’image de l’après-guerre, la France doit sortir de la dépendance américaine et chinoise. Seule la planification, autant démocratique qu’écologique est en mesure de relever ces défis immenses.

Il s'agit d’affronter des adversaires déterminés et puissants. Les moyens financiers des géants du secteurs sont immenses, ils disposent de relais puissants au sein des institutions de l'Union européenne et à la technostructure française qui a théorisé et construit son impuissance. À l’image d'un nécessaire retour de l’État planificateur et stratège\cite{interet2020servicespublics}, cette ambition collective doit passer par l'intégration du temps long et une vision globale de stratégies industrielles incluant les dimensions climatiques et écologiques. 
 
Cette approche est aux antipodes d’un discours basé sur la « start-up Nation ». Une start-up est par nature une jeune entreprise visant une création de valeur actionnariale sur des objectifs de marché identifiables voire court-termistes. Elle se cantonne le plus souvent à des produits précis sans vision globale ni apport structurant. Il ne s'agit pas ici de décourager l'innovation ou de nier l’ambition technologique de certains projets. Mais c’est une erreur funeste de faire reposer une stratégie sur cette approche minimale pour une puissance publique industrielle souveraine comme la France.

En conclusion, les débats concernant le déploiement de la 5G questionnent avec pertinence la désirabilité du progrès technique. Toute nouveauté n’est pas bonne en soit. Les crises écologiques ont déjà commencé et toute planification industrielle se doit d’intégrer les contraintes climatiques. L’empreinte carbone du numérique doit diminuer, et rapidement,  ce qui suppose de choisir et hiérarchiser les innovations, quitte à en refuser certaines. Plutôt que de s’aventurer pour une technologie à l’utilité discutable, il est plus important de raccorder l’ensemble de la population à un réseau internet stable et de qualité. L’accès à la fibre généralisé, opéré par un opérateur public est d’ailleurs une revendication de l’Electronic Frontier Fondation , portée jusqu’au plan de relance américain. Il ne s’agit donc pas de borner le progrès technique, mais de borner les droits et conséquences du marché. L’emprise du marché et des grands groupes privés sur la technique et nos moyens de communication doit cesser. Ces bornes s’appellent la démocratie et l’écologie.

\bibliographystyle{ieeetr}
\bibliography{document}
\end{document}
